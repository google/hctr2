% Copyright 2021 Google LLC
%
% Use of this source code is governed by an MIT-style
% license that can be found in the LICENSE file or at
% https://opensource.org/licenses/MIT.

%!TeX spellcheck = en-US

\documentclass[letterpaper,11pt]{article}

\usepackage{floatrow}
\usepackage{algpseudocode}
\usepackage{amsmath}
\usepackage[style=alphabetic,backend=biber,alldates=ymd]{biblatex}
\usepackage[logic,probability,advantage,adversary,landau,sets,operators]{cryptocode}
\usepackage{tikz}

\usetikzlibrary{groupops}

\newcommand*{\parintro}[1]{\textbf{#1}:}
\newcommand*{\Concat}{\Vert}
\newcommand*{\MM}{\mathit{MM}}
\renewcommand*{\CC}{\mathit{CC}}

\DeclareMathOperator{\XCTR}{XCTR}
\DeclareMathOperator{\GF}{GF}
\DeclareMathOperator{\fromint}{bin}
\DeclareMathOperator{\intify}{int}
\DeclareMathOperator{\pad}{pad}
\DeclareMathOperator{\Poly}{Poly}
\DeclareMathOperator{\HCTR}{HCTR2}

\title{HCTR2}

\begin{document}
\maketitle

\begin{abstract}
HCTR2
\end{abstract}

\section{Specification}
\begin{figure}
    % Copyright 2021 Google LLC
%
% Use of this source code is governed by an MIT-style
% license that can be found in the LICENSE file or at
% https://opensource.org/licenses/MIT.

\begin{tikzpicture}
  [line width=0.6,
   box/.style = {
     draw, rectangle, rounded corners
   },
   wire/.style = {
     ->,rounded corners=1.5pt
   },
   cbox/.style={
     rectangle,
     thick,
     draw,
     minimum height=1cm,
     text centered,
     anchor=center,
     rounded corners=2pt,
   },
   minimum size=0.5cm,
   x=1.5cm, y=1.5cm
  ]

  %Draw nodes
  \node[rectangle] at (0,0) (M) {$M$};
  \node[rectangle] at (3,0) (N) {$N$};
  \node[cbox] at (1,-1) (H1) {$H_h$};
  \node[XOR] at (M |- H1) (X1) {};
  \node[cbox] at (0,-3) (E) {$E_K$};
  \node[rectangle] at (4,-3) (T) {$T$};
  \node[XOR] at (0.75, -3) (XS) {};
  \node[cbox] at (2,-3) (CTR) {$\XCTR_K$};
  \node[XOR] at (N |- CTR) (X2) {};
  \node[cbox] at (1,-5) (H3) {$H_h$};
  \node[XOR] at (M |- H3) (X3) {};
  \node[rectangle] at (0,-6) (C) {$C$};
  \node[rectangle] at (X2 |- C) (D) {$D$};


  \draw[wire] (M) -- (X1);
  \draw[wire] (N) -- (X2);
  \draw[wire] ([yshift=-1ex] X2 |- H1) -- ([yshift=-1ex] H1.east);
  \draw[wire] (H1) -- (X1);
  \draw[wire] (X1) -- node[left] {\(\MM\)} (E);
  \draw[wire] (0,-2) -| (XS);
  \draw[wire] (XS) -- node[above] {\(S\)} (CTR);
  \draw[wire] (CTR) -- (X2);
  \draw[wire] (0,-4) -| (XS);
  \draw[wire] (E) -- node[left] {\(\CC\)} (X3);
  \draw[wire] (T) -- ++(-0.5, 0) |- ([yshift=+1ex] H1.east);
  \draw[wire] (T) -- ++(-0.5, 0) |- ([yshift=+1ex] H3.east);
  \draw[wire] ([yshift=-1ex] X2 |- H3) -- ([yshift=-1ex] H3.east);
  \draw[wire] (H3) -- (X3);
  \draw[wire] (X3) -- (C);
  \draw[wire] (X2) -- (D);
\end{tikzpicture}

    \caption{HCTR2}\label{fig}
\end{figure}
\begin{figure}
    \begin{floatrow}
        \ffigbox{
            \begin{algorithmic}[0]
                \Procedure{\(HCTR2_{k, T}\)}{$P$}
                \State $h \gets E_k(0^n)$
                \State $M \Concat N \gets P$, $|M| = n$
                \State $\MM \gets M \xor H_h(T, N)$
                \State $\CC \gets E_k(\MM)$
                \State $S \gets \MM \xor \CC$
                \State $D \gets N \xor \XCTR_k(S)[0;\abs{N}]$
                \State $C \gets \CC \xor H_h(T, D)$
                \State $R \gets C \Concat D$
                \State \textbf{return} $R$
                \EndProcedure
            \end{algorithmic}
        }{
            \caption{HCTR2 encryption}
        }
        \ffigbox{
            \begin{algorithmic}[0]
                \Procedure{\(HCTR2_{k, T}^{-1}\)}{$R$}
                \State $h \gets E_k(0^n)$
                \State $C \Concat D \gets R$, $|C| = n$
                \State $\CC \gets C \xor H_h(T, D)$
                \State $\MM \gets E_k(\CC)$
                \State $S \gets \MM \xor \CC$
                \State $N \gets D \xor \XCTR_k(S)[0;\abs{D}]$
                \State $M \gets \MM \xor H_h(T, N)$
                \State $P \gets M \Concat N$
                \State \textbf{return} $P$
                \EndProcedure
            \end{algorithmic}
        }{\caption{HCTR2 decryption}\label{pseudocode}}
    \end{floatrow}
\end{figure}

\parintro{Notation}
Subscripts may denote partial application; if we define $f: A \times B \rightarrow C$ and
$a \in A$ then $f_a: B \rightarrow C$, and if $f_a^{-1}$ exists then $f_a^{-1}(f_a(b)) = b$.
\begin{itemize}
    \item $\abs{X}$: length of $X \in \bin^{*}$ in bits
    \item $\lambda$: the empty string $\abs{\lambda} = 0$
    \item $\Concat$: bitstring concatenation
    \item $Y[a;l]$: the subsequence of $Y$ of length $l$ starting at the 0-based index $a$.
    \item $\pad_l(X) = X \Concat 0^v$
    where $v$ is the least integer $\geq 0$ such that $l$ divides $\abs{X} + v$
    \item $\intify: \bin^{*} \rightarrow \ZZ$: the
    standard little-endian map such that
    $\intify(\lambda) = 0$, $\intify(0 \Concat X) = 2\intify(X)$, $\intify(1 \Concat X) = 1 + 2\intify(X)$
    \item $\fromint_l(y) = X$: the unique
    $l$-bit sequence such that $\intify(X) \equiv y \pmod{2^l}$
    \item \(\oplus\)
    \item \(\odot\)
\end{itemize}
We map bytes to bitstrings with \(\fromint_8\).
Where we have eg \(P_0 \Concat \ldots \Concat P_{m-1} \gets P\), 
\(P_0, \ldots, P_{m-1}\) is the unique sequence of \(\ceil{\abs{P}/n}\) blocks
such that \(|P_i| = n\) for \(0 \leq i < m-1\).

Where \(|M| \equiv 0 \pmod n\) we define polynomial evaluation as \(\Poly_h(\lambda) = 0\), \(\Poly_h(M \Concat T) = h \odot (\Poly_h(M) \xor T)\) where \(|T| = n\). This is insensitive to prefixed zeroes \(\Poly_h(0^n \concat M) = \Poly_h(M)\) and linear in the second argument \(\Poly_h(M) \xor \Poly_h(M') = \Poly_h(M \xor M')\).

Given a hashing key \(h\) and a message \(M_1 \Concat M_2 \Concat \ldots \Concat M_l\) with \(\abs{M_i} = n\) for \(i < l\) and \(0 < \abs{M_i} \leq n\) for \(i = l\)

\section{Security bound}

We prove the security of HCTR2 using Patarin's H-coefficient technique. 

We require four properties from our hash function \(H_h\):

\printbibliography[heading=bibintoc]

\end{document}
