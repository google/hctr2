% Copyright 2021 Google LLC
%
% Use of this source code is governed by an MIT-style
% license that can be found in the LICENSE file or at
% https://opensource.org/licenses/MIT.
%!TeX spellcheck = en-US
\documentclass[letterpaper,11pt]{article}
\usepackage{floatrow}
\usepackage{algpseudocode}
\usepackage{amsmath}
\usepackage[style=alphabetic,backend=biber,alldates=ymd]{biblatex}
\usepackage[logic,probability,advantage,adversary,landau,sets,operators]{cryptocode}
\usepackage{tikz}
\usetikzlibrary{groupops}
\newcommand*{\parintro}[1]{\textbf{#1}:}
\newcommand*{\Concat}{\Vert}
\newcommand*{\MM}{\mathit{MM}}
\renewcommand*{\CC}{\mathit{CC}}
\DeclareMathOperator{\XCTR}{XCTR}
\DeclareMathOperator{\GF}{GF}
\DeclareMathOperator{\fromint}{bin}
\DeclareMathOperator{\intify}{int}
\DeclareMathOperator{\pad}{pad}
\DeclareMathOperator{\trunc}{trunc}
\DeclareMathOperator{\Poly}{Poly}
\DeclareMathOperator{\HCTR}{HCTR2}
\title{HCTR2}
\begin{document}
\maketitle
\begin{abstract}
\end{abstract}
\section{Specification}
\parintro{Notation}
Subscripts may denote partial application; if we define $f: A \times B \rightarrow C$ and
$a \in A$ then $f_a: B \rightarrow C$, and if $f_a^{-1}$ exists then $f_a^{-1}(f_a(b)) = b$.
\begin{itemize}
    \item $\abs{X}$: length of $X \in \bin^{*}$ in bits
    \item $\lambda$: the empty string $\abs{\lambda} = 0$
    \item $\Concat$: bitstring concatenation
    \item $Y[a;l]$: the subsequence of $Y$ of length $l$ starting at the 0-based index $a$.
    \item $\pad_l(X) = X \Concat 0^v$
    where $v$ is the least integer $\geq 0$ such that $l$ divides $\abs{X} + v$
    \item $\intify: \bin^{*} \rightarrow \ZZ$: the
    standard little-endian map such that
    $\intify(\lambda) = 0$, $\intify(0 \Concat X) = 2\intify(X)$, $\intify(1 \Concat X) = 1 + 2\intify(X)$
    \item $\fromint_l(y) = X$: the unique
    $l$-bit sequence such that $\intify(X) \equiv y \pmod{2^l}$
    \item $trunc_l(X)$ - truncation of $X \in \{0,1\}^{\ge l}$ to be exactly $l$ bits long.
    \item \(\oplus\)
    \item \(\odot\)
\end{itemize}
We map bytes to bitstrings with \(\fromint_8\).
Where we have eg \(P_0 \Concat \ldots \Concat P_{m-1} \gets P\), 
\(P_0, \ldots, P_{m-1}\) is the unique sequence of \(\ceil{\abs{P}/n}\) blocks
such that \(|P_i| = n\) for \(0 \leq i < m-1\).
Where \(|M| \equiv 0 \pmod n\) we define polynomial evaluation as \(\Poly_h(\lambda) = 0\), \(\Poly_h(M \Concat T) = h \odot (\Poly_h(M) \xor T)\) where \(|T| = n\). This is insensitive to prefixed zeroes \(\Poly_h(0^n \concat M) = \Poly_h(M)\) and linear in the second argument \(\Poly_h(M) \xor \Poly_h(M') = \Poly_h(M \xor M')\).
Given a hashing key \(h\) and a message \(M_1 \Concat M_2 \Concat \ldots \Concat M_l\) with \(\abs{M_i} = n\) for \(i < l\) and \(0 < \abs{M_i} \leq n\) for \(i = l\)
\begin{figure}
	\begin{floatrow}
		% Copyright 2021 Google LLC
%
% Use of this source code is governed by an MIT-style
% license that can be found in the LICENSE file or at
% https://opensource.org/licenses/MIT.
\begin{tikzpicture}
	[line width=0.6,
		box/.style = {
			draw, rectangle, rounded corners
		},
		wire/.style = {
			->,rounded corners=1.5pt
		},
		minimum size=0.5cm,
		x=1.5cm, y=1.5cm
	]
	%Draw nodes
	\node[rectangle] at (0,0) (S1) {$S \oplus 1$};
	\node[rectangle] at (1,0) (S2) {$S \oplus 2$};
	\node[rectangle] at (2,0) (S3) {$S \oplus 3$};
	\node[rectangle] at (2.5,0) (S5) {$\cdots$};

	\node[rectangle] at (0,-1) (E1) {$Enc_K$};
	\node[rectangle] at (1,-1) (E2) {$Enc_K$};
	\node[rectangle] at (2,-1) (E3) {$Enc_K$};
	\node[rectangle] at (2.5,-1) (E5) {$\cdots$};

	\node[rectangle] at (-0.5,-2) (P0) {$P_0$};
	\node[rectangle] at (0.5,-2) (P1) {$P_1$};
	\node[rectangle] at (1.5,-2) (P2) {$P_2$};

	\node[XOR] at (0, -2) (X1) {};
	\node[XOR] at (1, -2) (X2) {};
	\node[XOR] at (2, -2) (X3) {};
	\node[rectangle] at (2.5, -2) (X5) {$\cdots$};

	\node[rectangle] at (0,-3) (C0) {$C_0$};
	\node[rectangle] at (1,-3) (C1) {$C_1$};
	\node[rectangle] at (2,-3) (C2) {$C_2$};
	\node[rectangle] at (2.5,-3) (C4) {$\cdots$};

	\draw[wire] (S1) -- (E1) -- (X1) -- (C0);
	\draw[wire] (P0) -- (X1);
	\draw[wire] (S2) -- (E2) -- (X2) -- (C1);
	\draw[wire] (P1) -- (X2);
	\draw[wire] (S3) -- (E3) -- (X3) -- (C2);
	\draw[wire] (P2) -- (X3);
\end{tikzpicture}

		\caption{Modified CTR}\label{ctrfig}
	\ffigbox{
		\begin{algorithmic}[0]
			\Procedure{$Ctr_S$}{$M$}
			\State $M_0 \Concat \cdots \Concat \pad_l(M_{n-1})$ \gets M
			\For{$i \in \{0, \cdots, n-1\}$}
				\State $S_i \gets Enc_{E_K}(S \oplus (i + 1))$
				\State $C_i \gets S_i \oplus M_i$
			\EndFor
			\State \textbf{return} $\trunc_{|M|}(C_0 \Concat \cdots \Concat C_n)$
			\EndProcedure
		\end{algorithmic}
		}{
	\caption{
		Pseudocode for Modified CTR}\label{ctrpseudocode}
	}
	\end{floatrow}
\end{figure}

\subsection{Polynomial Hash Function}
HCTR2 uses a variation of the original HCTR polynomial hash function. Let $t$ be the number of padded blocks in $T$ and $n$ be the number of padded blocks in $X$.
\begin{align*}
	H_h(X, T) = T_0h^{t+n+2} + \cdots + T_th^{n+3} + X_0h^{n+2} + \cdots + X_nh^3 + (|T|+1)h^2 + (|X|+1)h
\end{align*}
The security proof of HCTR2 requires $H$ to have the following properties:

\begin{figure}
    % Copyright 2021 Google LLC
%
% Use of this source code is governed by an MIT-style
% license that can be found in the LICENSE file or at
% https://opensource.org/licenses/MIT.

%!BIB program = biber
%!TeX program = lualatex
%!TeX spellcheck = en-US

\documentclass[hctr2.tex]{subfiles}
\begin{document}
\begin{figure}
  \begin{tikzpicture}
    [thick,
     box/.style = {
       draw, rectangle, rounded corners
     },
     wire/.style = {
       ->,rounded corners=4pt
     },
     cbox/.style={
       rectangle,
       thick,
       draw,
       minimum height=1cm,
       text centered,
       anchor=center,
       rounded corners=2pt,
     },
     minimum size=0.4cm,
     x=0.4cm, y=0.4cm
    ]
    %Draw nodes
    \node[rectangle] at (0,0) (M) {$M$};
    \node[rectangle] at (13,0) (N) {$N$};
    \node[cbox] at (9,-3) (H1) {$H_{\hgen}$};
    \node[XOR] at (M |- H1) (X1) {};
    \node[cbox] at (0,-8) (E) {$E_k$};
    \node[rectangle] at (17,-8) (T) {$T$};
    \node[XOR] at (4, -8) (XS) {};
    \node[rectangle] at (2, -8) (L) {\(L\)};
    \node[cbox] at (H1 |- L) (XCTR) {$\XCTR_k$};
    \node[XOR] at (N |- XCTR) (X2) {};
    \node[cbox] at (9,-13) (H3) {$H_{\hgen}$};
    \node[XOR] at (M |- H3) (X3) {};
    \node[rectangle] at (0,-16) (U) {$U$};
    \node[rectangle] at (X2 |- U) (V) {$V$};
    \draw[wire] (M) -- (X1);
    \draw[wire] (N) -- (X2);
    \draw[wire] ([yshift=-1ex] X2 |- H1) -- ([yshift=-1ex] H1.east);
    \draw[wire] (H1) -- (X1);
    \draw[wire] (X1) -- node[left] {\(\MM\)} (E);
    \draw[wire] (0,-5) -| (XS);
    \draw[wire] (L) -- (XS);
    \draw[wire] (XS) -- node[above] {\(S\)} (XCTR);
    \draw[wire] (XCTR) -- (X2);
    \draw[wire] (0,-11) -| (XS);
    \draw[wire] (E) -- node[left] {\(\UU\)} (X3);
    \draw[wire] (T) -- ++(-2, 0) |- ([yshift=+1ex] H1.east);
    \draw[wire] (T) -- ++(-2, 0) |- ([yshift=+1ex] H3.east);
    \draw[wire] ([yshift=-1ex] X2 |- H3) -- ([yshift=-1ex] H3.east);
    \draw[wire] (H3) -- (X3);
    \draw[wire] (X3) -- (U);
    \draw[wire] (X2) -- (V);
  \end{tikzpicture}
  \caption{HCTR2}\label{hctr2fig}
\end{figure}
\end{document}

    \caption{HCTR2}\label{hctr2fig}
\end{figure}
\begin{figure}
    \begin{floatrow}
        \ffigbox{
            \begin{algorithmic}[0]
                \Procedure{\(HCTR2_{k, T}\)}{$P$}
                \State $h \gets E_k(0^n)$
                \State $M \Concat N \gets P$, $|M| = n$
                \State $\MM \gets M \xor H_h(T, N)$
                \State $\CC \gets E_k(\MM)$
                \State $S \gets \MM \xor \CC$
                \State $D \gets N \xor \XCTR_k(S)[0;\abs{N}]$
                \State $C \gets \CC \xor H_h(T, D)$
                \State $R \gets C \Concat D$
                \State \textbf{return} $R$
                \EndProcedure
            \end{algorithmic}
        }{
            \caption{HCTR2 encryption}
        }
        \ffigbox{
            \begin{algorithmic}[0]
                \Procedure{\(HCTR2_{k, T}^{-1}\)}{$R$}
                \State $h \gets E_k(0^n)$
                \State $C \Concat D \gets R$, $|C| = n$
                \State $\CC \gets C \xor H_h(T, D)$
                \State $\MM \gets E_k(\CC)$
                \State $S \gets \MM \xor \CC$
                \State $N \gets D \xor \XCTR_k(S)[0;\abs{D}]$
                \State $M \gets \MM \xor H_h(T, N)$
                \State $P \gets M \Concat N$
                \State \textbf{return} $P$
                \EndProcedure
            \end{algorithmic}
        }{\caption{HCTR2 decryption}\label{pseudocode}}
    \end{floatrow}
\end{figure}
\section{Security bound}
We prove the security of HCTR2 using Patarin's H-coefficient technique. 
We require four properties from our hash function \(H_h\):
\printbibliography[heading=bibintoc]
\end{document}
