% Copyright 2021 Google LLC
%
% Use of this source code is governed by an MIT-style
% license that can be found in the LICENSE file or at
% https://opensource.org/licenses/MIT.

%!BIB program = biber
%!TeX program = lualatex
%!TeX spellcheck = en-US

\documentclass[hctr2.tex]{subfiles}
\begin{document}
\subsection{Main lemma}\label{mainlemma}
In what follows, we take world X (the ``real world'') 
to be \(\HCTR[\Perm(n)]\),
ie HCTR with all calls to the block cipher
replaced with calls to a random permutation,
and world Y (the ``ideal world'') to be \(\pm\widetilde{\mathrm{rnd}}\),
which maps every query to a random response such that
all responses of the appropriate length are equally likely;
we then use the H-coefficient technique to bound
the distinguishing advantage between them
for a fixed adversary \(A \in \mathcal{A}(q, \sigma, t)\)
as defined in \autoref{definitions}.

We use superscripts to distinguish between queries;
where we refer to \(r\), \(s\), we assume that \(r < s\).
An encryption query \((T^s, P^s)\) yields ciphertext \(C^s\)
while a decryption query \((T^s, C^s)\)
yields plaintext \(P^s\).
We forbid ``pointless queries'':
encryption queries \((T^s, P^s)\)
such that \((T^r, P^r) = (T^s, P^s)\) for some \(r < s\), 
or decryption queries \((T^s, C^s)\)
such that \((T^r, C^r) = (T^s, C^s)\) for some \(r < s\),
whether query \(r\) was an encryption or decryption query.

For each query \(s\), 
let \(m^s \defeq \ceil{\abs{P^s}/n} = \ceil{\abs{C^s}/n}\)
be the number of blocks in the response,
and let \(d^s \defeq m^s + \ceil{\abs{T^s}/n}\)
be the number of blocks in the query.
Note that \(d^s\) is the degree of the hash
function polynomial used in query \(s\)
(since all but one block of the message is hashed),
and that \(\sum_s d^s \leq \sigma\).
We give the adversary some extra information
which is included in the transcript.
In world X, this information is:
\begin{itemize}
    \item the ``leftover block'' for each query:
    where a query has plaintext/ciphertext
    that is not a multiple of the block size,
    this is the extra output from the last
    block cipher call that is not used.
    For query \(s\), this is 
    \begin{displaymath}
        D^s = \XCTR_\pi(S^s)[\abs{P^s} - n; nm^s-\abs{P^s}]
    \end{displaymath}
    \item the hash key \(\hgen\), given after all queries are complete
    \item the mask \(L\), given after all queries are complete
\end{itemize}
In world Y, random output of the expected
length is substituted. Since the adversary can always ignore
this information, giving it to them cannot make their performance worse.

\subsubsection{Good and bad transcripts}
For \(j \in \{1 \ldots m^s-1\}\) we define \(S_j^s = S^s \xor \fromint(j)\),
the block cipher inputs used in XCTR, and \(Y_j^s\) the corresponding
outputs, so that in world X \(Y_j^s = \pi(S_j^s)\) and 
\(Y_1^s \Concat \cdots \Concat Y_{m^s-1}^s = \XCTR_\pi(S^s)[0;n(m^s -1)]\).

Given the full transcript, including \(\hgen\) and \(L\),
we can infer all block cipher plaintexts and ciphertexts.
For each query
(omitting the query superscript \(s\) for readability)
we know \(T\), \(P\), \(C\) and \(D\) and so can infer:
\begin{align*}
    M \Concat N &= P \\
    U \Concat V &= C \\
    \MM &= M \xor H_{\hgen}(T, N)\\
    \UU &= U \xor H_{\hgen}(T, V)\\
    S &= \MM \xor \UU \xor L\\ 
    S_j &= S \xor \fromint(j)\\
    Y_1 \Concat \cdots \Concat Y_{m-1} &= (N \xor V) \Concat D
\end{align*}
This gives us multisets \(\mathcal{D}\) and \(\mathcal{R}\)
of inferred block cipher plaintexts and ciphertexts:
\begin{align*}
    \mathcal{D} \defeq [\fromint(0), \fromint(1)] \uplus &
    \biguplus_s [\MM^s, S_1^s, \ldots, S_{m^s-1}^s]\\
    \mathcal{R} \defeq [\hgen, L] \uplus &
    \biguplus_s [\UU^s, Y_1^s, \ldots, Y_{m^s-1}^s]
\end{align*}
We therefore infer
\(\abs{\mathcal{D}} = \abs{\mathcal{R}} = \sigma_m \defeq 2 + \sum_s m^s\)
block cipher plaintexts/ciphertexts
from \(\sigma_m n\) bits of response
(including the extra information \(D^s\), \(\hgen\), \(L\)).
A transcript is ``bad'' (\(\tau \in \Tb\))
iff any entry in \(\mathcal{D}\) or \(\mathcal{R}\)
has multiplicity greater than one,
ie if any pair of inferred plaintexts are the same, or
if any pair of inferred ciphertexts are the same.

Since responses in world Y are coin flips,
the probability of a 
particular \(\tau \in \Tc\), good or bad,
in world Y is always simply \(2^{-\sigma_m n}\).
For a transcript \(\tau \in \Tg\),
the probability in world X
is the probability of all of those plaintext/ciphertext
pairs being part of a given random permutation.
This is 
\(\prod_{i=0}^{\sigma_m -1}1/(2^n - i)\).
Thus \(\prob{Y = \tau} \leq \prob{X = \tau}\)
for all \(\tau \in \Tg\) as required (\autoref{hco}).

\subsubsection{Case analysis of collisions}
Next we bound \(\prob{Y \in \Tb}\).
We consider a case by case analysis of possible collisions,
in either inferred plaintexts (\(\mathcal{D}\))
or inferred ciphertexts (\(\mathcal{R}\)),
and bound the probability in world Y each case.

Responses are random in world Y,
but some caution is required.
If we know the adversary's query \(s\), then conditioning on that,
we cannot treat the response to query \(r < s\) as uniformly random;
if the choice of later query depends on the earlier response,
knowing the later query is information about the earlier response.
However, conditioning on a query and all prior queries and responses,
we still have that \(\hgen\), \(L\), and the
query response are uniformly random and independent,
and so we can consider them in any order.

Consider for example the case \(S_i^r \qeq \MM^s\):
for a given \(r\), \(s\), \(i\),
we want to evaluate \(\prob{S_i^r = \MM^s}\).
From our inference rules, \(S_i^r = \MM^s\)
iff \(L = M^r \xor H_{\hgen}(T^r, N^r) \xor U^r \xor H_{\hgen}(T^r, V^r)
\xor \fromint(i) \xor M^s \xor H_{\hgen}(T^s, N^s)\).
\(\hgen\) and \(L\) are given at the end of the transcript
and so are independent of all other queries and responses.
If we knew the entire transcript except \(L\),
we would know the entire right hand side of this equation.
In world X, we would also know for example that \(L \neq \hgen\), but
in world Y, conditioning on the rest of the transcript,
all values of \(L\) are equally likely;
therefore this equation holds with probability exactly \(1/2^n\).

There are twenty-two cases to consider,
ten of which arise because we use the block cipher
to generate \(\hgen\) and \(L\).
In fourteen cases, the probability of a collision
between two specific blocks is \(1/2^n\):

\subfile{casetable.tex}
\begin{itemize}
    \item
    \(\hgen \qeq L\),
    \(L \qeq \UU^s\),
    \(L \qeq Y_j^s\),
    \(\fromint(0) \qeq S_j^s\),
    \(\fromint(1) \qeq S_j^s\),
    \(S_i^r \qeq \MM^s\),
    \(\MM^r \qeq S_j^s\),
    \(\MM^s \qeq S_j^s\):
    Given \(\hgen\) and all queries and responses,
    there is exactly one value of \(L\)
    which causes the equation to hold.
    \item
    \(\hgen \qeq Y_j^s\),
    \(\UU^r \qeq Y_j^s\),
    \(Y_i^r \qeq Y_j^s\),
    \(Y_i^s \qeq Y_j^s\) where \(i < j\): 
    If query \(s\) is an encryption query,
    then given \(\hgen\), query \(s\), 
    all prior queries and responses, and
    \(C^s[0;jn]\), there is exactly one value of
    \(C^s[jn;n]\) that results in the equation holding.
    If \(s\) is a decryption query, the same reasoning holds
    with \(P^s\), \(C^s\) swapped.
    \item 
    \(\UU^s \qeq Y_j^s\):
    If query \(s\) is a decryption query,
    the exact argument above applies.
    If it is an encryption query,
    then given \(\hgen\), \(T^s\), \(P^s\),
    and \(V^s\), there is exactly
    one value of \(U^s\) that results
    in the equation holding.
    \item 
    \(S_i^r \qeq S_j^s\):
    If query \(s\) is an encryption query,
    then given \(\hgen\), \(L\), \(T^s\),
    \(P^s\), \(V^s\)
    and all prior queries and responses,
    there is exactly one response value \(U^s\)
    that results in the equation holding.
    For a decryption query, 
    given \(\hgen\), \(L\), \(T^s\),
    \(C^s\), \(N^s\)
    and all prior queries and responses,
    there is exactly one response value \(M^s\)
    that results in the equation holding.
\end{itemize}

In two cases, a collision is impossible:

\begin{itemize}
    \item \(\fromint(0) \qeq \fromint(1)\): 
    Trivially impossible.
    \item 
    \(S_i^s \qeq S_j^s\):
    This is impossible; 
    \(S_i^s \xor S_j^s = \fromint(i) \xor \fromint(j)\).
\end{itemize}

There are six cases where the probability may
be greater than \(1/2^n\).
Considering first collisions with \(\MM^s\) where
query \(s\) is an encryption query:

\begin{itemize}
    \item \(\MM^r \qeq \MM^s\):
    This holds iff \(M^r \xor H_{\hgen}(T^r, N^r) = M^s \xor H_{\hgen}(T^s, N^s)\);
    by hash function property 2, this holds with probability at most
    \(\max(d^r, d^s)/2^n\).
    \item \(\fromint(0) \qeq \MM^s\):
    This holds iff \(\fromint(0) = M^s \xor H_{\hgen}(T^s, N^s)\);
    by hash function property 1, this holds with probability at most
    \(d^s/2^n\).
    \item \(\fromint(1) \qeq \MM^s\):
    As above.
\end{itemize}

In each case, if query \(s\) is a decryption query,
then given \(\hgen\), \(T^s\), \(C^s\), \(N^s\),
and all prior queries and responses,
all values of \(M^s\) are equally likely 
and a single value causes the equation to hold,
for a probability of \(1/2^n\).

Similarly, considering collisions with \(\UU^s\) where
query \(s\) is a decryption query:

\begin{itemize}
    \item \(\UU^r \qeq \UU^s\): 
    This holds iff \(U^r \xor H_{\hgen}(T^r, V^r) = U^s \xor H_{\hgen}(T^s, V^s)\);
    by hash function property 2, this holds with probability at most
    \(\max(d^r, d^s)/2^n\).
    \item \(Y_i^r \qeq \UU^s\):
    This holds iff \(Y_i^r = U^s \xor H_{\hgen}(T^s, V^s)\);
    by hash function property 1, this holds with probability at most
    \(d^s/2^n\).
    \item \(\hgen \qeq \UU^s\):
    This holds iff \(\hgen = U^s \xor H_{\hgen}(T^s, V^s)\);
    by hash function property 3, this holds with probability at most
    \(d^s/2^n\).
\end{itemize}

As above, if query \(s\) is an encryption query,
then given \(\hgen\), \(T^s\), \(P^s\), \(V^s\),
and all prior queries and responses,
all values of \(U^s\) are equally likely
and a single value causes the equation to hold,
for a probability of \(1/2^n\).

\autoref{domaincollision} illustrates the various cases
we consider for inferred block cipher plaintext collisions.
Rows represent the terms
on the left hand side of the collision, while
columns represent the terms on the right; 
for example, the top left box represents
\(\fromint(0) \qeq \fromint(1)\). Where a square is left blank
it is either because it represents
comparing a term to itself (eg \(\MM^s \qeq \MM^s\))
or because it represents something that
is already handled elsewhere
(eg considering \(\fromint(1) \qeq \MM^s\)
handles the \(\MM^r \qeq \fromint(1)\) case).
A square is colored red
and marked \(0\) if the
probability of a particular collision
of that kind is zero, grey
and marked \(1\) if
the probability is always \(1/2^n\),
and green if the probability may be
greater than \(1/2^n\) and depends
on the number of solutions to a
particular polynomial;
\(d^s\) when there are at most \(d^s\) solutions,
\(\max\) where there are at most
\(\max(d^r, d^s)\) solutions.
Even where a square is green,
if query \(s\) is a decryption query,
the probability of a particular collision of
that kind is \(1/2^n\).
\autoref{rangecollision} covers
block cipher ciphertext collisions; in this case,
it is only decryption queries where
probabilities may be above \(1/2^n\).

\subsubsection{Summing collision bounds}
To establish an upper bound on the probability that any pair will collide,
we sum collision probabilities
for all pairs in the multiset \(\mathcal{D}\)
and all pairs in the multiset \(\mathcal{R}\).
To make summing easier, we define a ``correction'' \(c\):
\begin{align*}
    & \prob{Y \in \Tb} \\
    = &\prob{\exists [a, b] \subseteq \mathcal{D}: a = b \vee \exists [a, b] \subseteq \mathcal{R} : a = b} \\
    \leq &
        \Bigg(\sum_{[a, b] \subseteq \mathcal{D}} \prob{a = b}\Bigg)
      + \Bigg(\sum_{[a, b] \subseteq \mathcal{R}} \prob{a = b}\Bigg)
      \\
    = &\left.\left(2\binom{\sigma_m}{2} + c
    \right)\middle/2^n\right. \\
    \intertext{where}
    c \defeq &
    \Bigg(\sum_{[a, b] \subseteq \mathcal{D}} 2^n\prob{a = b} -1\Bigg)
    + \Bigg(\sum_{[a, b] \subseteq \mathcal{R}} 2^n\prob{a = b} -1\Bigg)
\end{align*}
This rearrangement is so that the fourteen cases above
which have a probability of colliding of \(1/2^n\)
make zero contribution to \(c\)
and so need not be considered further;
only the remaining eight
%(red and green boxes in \autoref{domaincollision} and \autoref{rangecollision})
need to be considered. Define
\(c = c_b + c_f + c_w + c_a\) where
\begin{itemize}
    \item \(c_b\) covers collisions within 
    \(\{\fromint(0), \fromint(1)\}\) and within \(\{\hgen, L\}\)
    \item \(c_f\) covers collisions
    for all \(s\)
    between \(\{\fromint(0), \fromint(1)\}\) and \(\{\MM^s, S_1^s, \ldots, S_{m^s-1}^s\}\)
    and between \(\{\hgen, L\}\) and \(\{\UU^s, Y_1^s, \ldots, Y_{m^s-1}^s\}\)
    \item \(c_w\) covers collisions
    for all \(s\)
    within \(\{\MM^s, S_1^s, \ldots, S_{m^s-1}^s\}\)
    and within \(\{\UU^s, Y_1^s, \ldots, Y_{m^s-1}^s\}\)
    \item \(c_a\) covers collisions
    for all \(r < s\)
    between \(\{\MM^r, S_1^r, \ldots, S_{m^r-1}^r\}\) and \(\{\MM^s, S_1^s, \ldots, S_{m^s-1}^s\}\)
    and between \(\{\UU^r, Y_1^r, \ldots, Y_{m^r-1}^r\}\) and \(\{\UU^s, Y_1^s, \ldots, Y_{m^s-1}^s\}\)
\end{itemize}

\(c_b = -1\), since 
\(\fromint(0) \qeq \fromint(1)\) is impossible.

For \(c_f\): if query \(s\) is an encryption query,
the only nonzero contributions come from the pairs
\(\fromint(0) \qeq \MM^s\) and
\(\fromint(1) \qeq \MM^s\). In each of these
cases the probability bound is not \(1/2^n\) but
\(d^s/2^n\), implying a correction of at most
\(2(d^s - 1)\) for each encryption query.
If query \(s\) is a decryption query,
we need only consider the pair \(\hgen \qeq \UU^s\)
for a correction of at most \(d^s - 1\). 
Summing across all queries, we conclude that
\begin{align*}
    c_f \leq &\sum_s \max(2(d^s - 1), d^s - 1) \\
        = &\sum_s 2(d^s - 1) \\
        \leq & 2\sigma
\end{align*}

For \(c_w\): we need only consider \(S_i^s \qeq S_j^s\),
which is impossible:
\begin{align*}
    c_w = &\sum_s -\binom{m^s -1}{2} \\
        \leq &0
\end{align*}

For \(c_a\): if query \(s\) is an encryption query,
\(\MM^r \qeq \MM^s\) gives a correction of
at most \(\max(d^r, d^s) -1\);
if it is a decryption query,
\(\UU^r \qeq \UU^s\) and \(Y_i^r \qeq \UU^s\)
give a correction of at most
\(\max(d^r, d^s) -1 + (m^r -1 )(d^s -1)\).
Summing across queries, we find
\begin{align*}
    c_a \leq &\sum_{r<s} \max(\max(d^r, d^s) -1, \max(d^r, d^s) -1 + (m^r -1 )(d^s -1)) \\
           = &\sum_{r<s} \max(d^r, d^s) -1 + (m^r -1 )(d^s -1) \\
        \leq & \sum_{r < s} d^r + d^s -1 + (m^r -1 )(d^s -1)\\
        \leq & (q-1)\sigma + \sum_{r < s} (m^r -1 )(d^s -1) \\
        \leq & (q-1)\sigma + \sum_{r < s} (d^r -1)(d^s -1) \\
        \leq & (q-1)\sigma + \binom{\sigma}{2}
\end{align*}

Applying the H-coefficient technique, we conclude that
\begin{align*}
    & \advantage{\pm\widetilde{\mathrm{rnd}}}{\HCTR[\Perm(n)]}[(q, \sigma, t)] \\
    \leq & \prob{Y \in \Tb} \\
    \leq & \left.\left(2\binom{\sigma_m}{2} + c_b + c_f + c_w + c_a\right)\middle/2^n\right. \\
    \leq & \left.\left(2\binom{\sigma_m}{2} - 1 + 2\sigma + (q-1)\sigma + \binom{\sigma}{2}\right)\middle/2^n\right. \\
    \leq & \left.\left(2\binom{\sigma + 2}{2} - 1 + 2\sigma + (q-1)\sigma + \binom{\sigma}{2}\right)\middle/2^n\right. \\
    = & \left.\left(2\left(\binom{\sigma}{2} + 2\sigma + 1\right) - 1 + 2\sigma + (q-1)\sigma + \binom{\sigma}{2}\right)\middle/2^n\right. \\
    = & \left.\left(3\sigma(\sigma - 1)/2 + q\sigma + 5\sigma + 1\right)\middle/2^n\right. \\
    = & \left.\left(3\sigma^2 + 2q\sigma + 7\sigma + 2\right)\middle/2^{n+1}\right.
\end{align*}
\end{document}
