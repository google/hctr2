% Copyright 2021 Google LLC
%
% Use of this source code is governed by an MIT-style
% license that can be found in the LICENSE file or at
% https://opensource.org/licenses/MIT.

%%% script AUTHOR: Paul Crowley, Nathan Huckleberry, Eric Biggers%%%
%%% script TITLE: Length-preserving encryption with {HCTR2} %%%

%!BIB program = biber
%!TeX program = lualatex
%!TeX spellcheck = en-US

\documentclass[journal=tosc,submission,floatrow]{iacrtrans}
\usepackage[none]{hyphenat}

\tolerance=9999

\usepackage{algpseudocode}
\usepackage{amsmath}
\usepackage[style=alphabetic,backend=biber,alldates=ymd]{biblatex}
\usepackage{bm}
\usepackage[logic,probability,advantage,adversary,landau,sets,operators]{cryptocode}
\usepackage{tikz}

\usetikzlibrary{groupops}

\author{Anonymous submission}
%\author{Paul~Crowley \and Nathan~Huckleberry \and Eric~Biggers}
%\institute{Google LLC \\ \email[paulcrowley@google.com,nhuck@google.com,ebiggers@google.com]{{paulcrowley,nhuck,ebiggers}@google.com}}

\addbibresource{bib.bib}

\newcommand*{\Concat}{\Vert}
\newcommand*{\defeq}{\stackrel{\text{def}}{=}}
\newcommand*{\hgen}{\bar{h}}
\newcommand*{\hpoly}{h}
\newcommand*{\MM}{\mathit{MM}}
\newcommand*{\qeq}{\stackrel{\text{?}}{=}}
\newcommand*{\Tb}{\mathcal{T}_\mathrm{bad}}
\newcommand*{\Tc}{\mathcal{T}_\mathrm{c}}
\newcommand*{\Tg}{\mathcal{T}_\mathrm{good}}
\newcommand*{\UU}{\mathit{UU}}

\DeclareMathOperator{\fromint}{bin}
\DeclareMathOperator{\GF}{GF}
\DeclareMathOperator{\HCTR}{HCTR2}
\DeclareMathOperator{\pad}{pad}
\DeclareMathOperator{\Perm}{Perm}
\DeclareMathOperator{\POLYVAL}{POLYVAL}
\DeclareMathOperator{\poly}{poly}
\DeclareMathOperator{\XCTR}{XCTR}

\title{Length-preserving encryption with HCTR2}

\usepackage{subfiles}
\begin{document}
\maketitle
\keywords{super-pseudorandom permutation \and
variable input length \and
tweakable encryption}

\begin{abstract}
On modern processors HCTR\cite{hctr} is
one of the most efficient constructions
for building a tweakable super-pseudorandom permutation. However,
a bug in the specification and another in
Chakraborty and Nandi's security proof\cite{hctrquad}
invalidate the claimed security bound. We here present HCTR2,
which fixes these issues and improves the
security bound, performance and flexibility.
%GitHub: \url{https://github.com/google/hctr2}
\end{abstract}

\subfile{introduction.tex}
\subfile{specification.tex}
\subfile{security.tex}
\subfile{hctrissues.tex}
\subfile{design.tex}
\subfile{implementation.tex}

\printbibliography[heading=bibintoc]
\appendix
\subfile{injective.tex}
%\subfile{polyvalimpl.tex}

\end{document}
