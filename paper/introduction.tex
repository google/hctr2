% Copyright 2021 Google LLC
%
% Use of this source code is governed by an MIT-style
% license that can be found in the LICENSE file or at
% https://opensource.org/licenses/MIT.

%!BIB program = biber
%!TeX program = lualatex
%!TeX spellcheck = en-US

\documentclass[hctr2.tex]{subfiles}
\begin{document}
\section{Introduction}

A \emph{tweakable super-pseudorandom permutation} (tweakable SPRP) is
a family of permutations indexed by tweak and input length, which
appear to be random permutations to an adversary without the key who
can make encryption and decryption queries\cite{cmc}. \cite{adiantum}
includes a detailed history of length-preserving encryption. A
tweakable SPRP is a highly general and flexible cryptographic
construction. One common use is in disk sector encryption: if the
ciphertext must be the same size as the plaintext, with no extra room
for nonce or MAC, a tweakable SPRP represents an upper bound on the
achievable security. If a variable-length tweak is accepted, it can
also serve as a nonce-misuse-resistant AEAD mode: concatenate the
nonce and the associated data to form the tweak, and authenticate the
message by appending zeroes to the plaintext which will be checked on
decryption\cite{encodethenencrypt, aez}.

\subfile{hctr2fig.tex}
In this paper, we present a specification (\autoref{specification})
and security bound (\autoref{security}) for HCTR2, a tweakable SPRP
based on HCTR\cite{hctr} and inheriting the following advantages:
\begin{itemize}
    \item simple
    \item efficient on modern processors, using a single block cipher
    invocation and two \(\GF(2^{128})\) multiplications per 16-byte
    block
    \item naturally handles ciphertext of any length of 16 bytes or greater
    \item tight quadratic security claim
\end{itemize}
HCTR2 addresses these issues in HCTR:
\begin{itemize}
    \item \cite{kumarhctr} observes that HCTR's hash function
    is not almost-XOR-universal\cite{eadu} as claimed (\autoref{badhash}).
    HCTR2's hash function fixes this property.
    \item Separately,
    an error in the proof presented in \cite{hctrquad}
    invalidates the quadratic security bound claimed in that paper (\autoref{badproof}); with our revised
    mode we can claim a tighter quadratic bound.
    \item HCTR2 supports using tweaks of any length with a single key.
    \item HCTR's key is a block cipher key, plus an \(n\)-bit
    hash key. HCTR2's key is simply
    the block cipher key.
    \item We modify the hash function to allow more precomputation
    for greater performance.
    \item We specify endianness and the like for interoperability.
    \item We provide a sample implementation and test vectors.
\end{itemize}
We discuss our design decisions (\autoref{design})
and report on its implementation on x86-64 and ARM64 (\autoref{implementation}).
We know of no patents affecting HCTR2.
\end{document}
