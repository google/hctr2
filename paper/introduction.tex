% Copyright 2021 Google LLC
%
% Use of this source code is governed by an MIT-style
% license that can be found in the LICENSE file or at
% https://opensource.org/licenses/MIT.

%!BIB program = biber
%!TeX program = lualatex
%!TeX spellcheck = en-US

\documentclass[hctr2.tex]{subfiles}
\begin{document}
\section{Introduction}
A symmetric encryption construction such as AES-GCM\cite{gcm}
will produce a ciphertext which is larger than
the plaintext, once the nonce and authentication tag are included. However,
there are situations such as disk sector encryption
where the ciphertext must be the same size as the
plaintext, and what is required is a tweakable
\emph{super-pseudorandom permutation} (SPRP):
a family of permutations indexed by tweak and input length,
which appear to be random permutations to an adversary without the key
who can make encryption and decryption queries\cite{cmc}.
\cite{adiantum} includes a detailed history of length-preserving encryption.
Where AES and multiplication in \(\GF(2^{128})\) are hardware-accelerated,
HCTR\cite{hctr} is among the most efficient 
tweakable SPRP constructions, using
a single block cipher invocation and two \(\GF(2^{128})\) multiplications
per 16-byte block.

In this paper, we present 
a specification (\autoref{specification})
and security bound (\autoref{security})
for a revised version, HCTR2, which 
addresses the following issues:
\begin{itemize}
    \item \cite{kumarhctr} observes that HCTR's hash function
    is not almost-XOR-universal\cite{eadu} as claimed (\autoref{badhash}).
    HCTR2's hash function fixes this property.
    \item Separately,
    an error in the proof presented in \cite{hctrquad}
    invalidates the security bound claimed in that paper (\autoref{badproof}); with our revised
    mode we can claim a tighter bound.
    \item HCTR2 supports using tweaks of any length with a single key.
    \item HCTR's key is a block cipher key, plus an \(n\)-bit
    hash key. HCTR2's key is simply
    the block cipher key.
    \item We modify the hash function to allow more precomputation
    for greater performance.
    \item We specify endianness and the like for interoperability.
    \item We provide a sample implementation and test vectors.
\end{itemize}
We discuss our design decisions (\autoref{design})
and report on its implementation on x86-64 and ARM64 (\autoref{implementation}).
We know of no patents affecting HCTR2.
\subfile{hctr2fig.tex}
\end{document}
