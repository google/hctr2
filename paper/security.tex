% Copyright 2021 Google LLC
%
% Use of this source code is governed by an MIT-style
% license that can be found in the LICENSE file or at
% https://opensource.org/licenses/MIT.

%!BIB program = biber
%!TeX program = lualatex
%!TeX spellcheck = en-US

\documentclass[hctr2.tex]{subfiles}
\begin{document}
\section{Security of HCTR2}\label{security}
Following the approach described in \cite{concrete},
we prove that if the underlying block cipher is secure,
then HCTR2 has good security properties.
The security bound proven appears in \autoref{securitybound}.
\subsection{Definitions}\label{definitions}
We use \(x \sample S\) to mean ``\(x\) is sampled from \(S\)
uniformly at random'', and we write
\(A^{\mathcal{O}, \mathcal{O}'}\Rightarrow 1\) to refer
to the event ``adversary \(A\),
given access to oracles \(\mathcal{O}\) and \(\mathcal{O}'\),
returns 1''.

Let $\Perm(n)$ denote the set of all permutations on \(\bin^n\).
Per \cite{concrete}, for a block cipher 
\(E: \mathcal{K} \times \bin^n \rightarrow \bin^n\)
the distinguishing advantage of an adversary \(A\) is:
%
\begin{align*}
    \advantage{\pm \mathrm{prp}}{E}[(A)] \defeq
    {}&\left\lvert\probsub{k \sample \mathcal{K}}{A^{E_k,E_k^{-1}}\Rightarrow 1}\right.
    \\
    {}&\left. - \probsub{\pi \sample \Perm(n)}{A^{\pi,\pi^{-1}}\Rightarrow 1}\right\rvert
    \\
    \intertext{Define}
    \advantage{\pm \mathrm{prp}}{E}[(q, t)] \defeq
    {}&\max_{A \in \mathcal{A}(q, t)} \advantage{\pm \mathrm{prp}}{E}[(A)]
\end{align*}
where $\mathcal{A}(q, t)$
is the set of all adversaries that make at most $q$ queries and take at most $t$ time.

Let $\Perm^\mathcal{T}(\mathcal{M})$
denote the set of all
tweakable length-preserving permutations
$\bm{\pi} : \mathcal{T} \times \mathcal{M} \rightarrow \mathcal{M}$
such that for all $T, M \in \mathcal{T} \times \mathcal{M}$,
$\abs{\bm{\pi}(T, M)} = \abs{M}$, and
for all $T \in \mathcal{T}$, $\bm{\pi}_{T}$ is a permutation on \(\mathcal{M}\).
In an abuse of notation
we use $\bm{\pi}^{-1}$ to refer to the function
such that $\bm{\pi}^{-1}(T, \bm{\pi}(T, M)) = M$ ie $(\bm{\pi}^{-1})_T = (\bm{\pi}_T)^{-1}$.

Per~\cite{cmc}, for a tweakable super-pseudorandom permutation
$\bm{E} : \mathcal{K} \times \mathcal{T} \times \mathcal{M} \rightarrow \mathcal{M}$
the distinguishing advantage of an adversary $A$ is:
%
\begin{align*}
    \advantage{\pm \widetilde{\mathrm{prp}}}{\bm{E}}[(A)] \defeq
    {}&\left\lvert\probsub{k \sample \mathcal{K}}{A^{\bm{E}_k,\bm{E}_k^{-1}}\Rightarrow 1}\right.
    \\
    {}&\left. - \probsub{\bm{\pi} \sample \Perm^\mathcal{T}(\mathcal{M})}
        {A^{\bm{\pi},\bm{\pi}^{-1}}\Rightarrow 1}\right\rvert
    \\
    \intertext{Define}
    \advantage{\pm \widetilde{\mathrm{prp}}}{\bm{E}}[(q, \sigma, t)]
    \defeq {}&
    \max_{A \in \mathcal{A}(q, \sigma, t)} \advantage{\pm \widetilde{\mathrm{prp}}}{\bm{E}}[(A)]
\end{align*}
where \(\mathcal{A}(q, \sigma, t)\)
is the set of all adversaries 
that make at most \(q\) queries
and take at most \(t\) time,
such that the total number of blocks sent in all queries is
at most \(\sigma\) ie
\begin{displaymath}
    \sum_s \ceil{\abs{T^s}/n} + \ceil{\abs{P^s}/n} \leq \sigma
\end{displaymath}
where \(\abs{T^s}, \abs{P^s}\) are the length of the tweak and the message presented in query \(s\).

We use \(\HCTR[\pi]\) to refer to HCTR2 in which invocation
of the block cipher is replaced with invocation of the
permutation \(\pi \in \Perm(n)\).
\(\XCTR_\pi\) refers to a similar substitution.
\(\HCTR[E]\) refers to HCTR2 using the block cipher \(E\),
ie \(\HCTR[E_k]\) for \(k \sample \mathcal{K}\), while
\(\HCTR[\Perm(n)]\) refers to \(\HCTR[\pi]\)
for \(\pi \sample \Perm(n)\).

\subsection{Hash function}\label{hproperties}
Define \(\poly(M)\) to refer to the formal polynomial
\(\poly(M_0 \Concat \cdots \Concat M_{l-1})
\defeq  M_0\hpoly^{l-1} \xor \cdots \xor M_{l-1}\).
While for example \(\hpoly + 2\) and \(2\hpoly + 1\) can be equal
in value if \(\hpoly = 1\),
they are not equal as formal polynomials;
two formal polynomials are only equal
if every coefficient is equal. Thus \(\poly(M) = \poly(M')\)
only if \(M = 0^{ln} \Concat M'\) for some \(l\) or vice versa.

Define \(H(T, M)\) as the formal polynomial in \(\hpoly\) given
by that tweak and message.
\(H_{\hgen}(T, M)\) is then evaluation of this polynomial at
\(\hpoly = x^{-n}\hgen\); POLYVAL evaluates at this point for
performance reasons.
\begin{align*}
    & H(T, M) \\
    \defeq &
    \begin{cases}
        \poly(\fromint(2\abs{T} + 2) \Concat \pad(T) \Concat M \Concat 0^n) &
        \text{if } n \text{ divides } \abs{M} \\
        \poly(\fromint(2\abs{T} + 3) \Concat \pad(T) \Concat \pad(M \Concat 1) \Concat 0^n) &
        \text{otherwise}
    \end{cases}
\end{align*}

We depend on the following properties of this map 
onto formal polynomials:
\begin{itemize}
    \item The map is injective
    \item The polynomial is never \(0\) or \(x^n\hpoly\)
    \item The constant term is always zero
    \item The polynomial is of degree at most
    \begin{displaymath}
        d(T, M) \defeq 1 + \ceil{\abs{T}/n} + \ceil{\abs{M}/n}
    \end{displaymath}
\end{itemize}

For the first property, see \autoref{injective}.
For the second, observe that \(H(T, M)\)
can be of degree 1 only if \(\abs{T} = \abs{M} = 0\),
in which case the polynomial is \(x\hpoly\).
Since \(x^{n-1} \neq 1\) we have that \(x^n \neq x\).

For any nonzero polynomial \(p(\hpoly)\)
in \(\GF(2^n)\), there are at most \(\deg(p)\) values \(\hpoly\)
such that \(p(\hpoly) = 0\), and therefore
\(\probsub{\hpoly\sample{\bin^n}}{p(\hpoly) = 0} \leq \deg(p)/2^n\).
Since multiplication by a nonzero field element
is a bijection of the field onto itself, it follows that
\(\probsub{\hgen\sample{\bin^n}}{p(x^{-n}\hgen) = 0} \leq \deg(p)/2^n\).
From this we infer three properties of \(H_{\hgen}(T, M)\):
\begin{description}
    \item[Property 1]
    For any \(T, M\) and any \(g \in \bin^n\),
    \begin{displaymath}
        \probsub{\hgen\sample{\bin^n}}{H_{\hgen}(T, M) = g} \leq d(T, M)/2^n
    \end{displaymath}
    Proof: since \(H(T, M)\) is nonzero and has
    a zero constant term,
    the polynomial \(H(T, M) \xor g\) 
    is nonzero and has the same degree, at most \(d(T, M)\).
    \item[Property 2] 
    For any \((T_1, M_1) \neq (T_2, M_2)\) and any \(g \in \bin^n\)
    \begin{align*}
        & \probsub{\hgen\sample{\bin^n}}{H_{\hgen}(T_1, M_1) \xor H_{\hgen}(T_2, M_2) = g} \\
        \leq  & \max(d(T_1, M_1), d(T_2, M_2))/2^n
    \end{align*}
    Proof: \(H\) is injective onto polynomials
    and the constant term is zero, therefore
    \(H(T_1, M_1) \xor H(T_2, M_2) \xor g\)
    is not the zero polynomial and
    has degree at most \(\max(d(T_1, M_1), d(T_2, M_2))\).
    This is the almost-XOR-universal property.
    \item[Property 3]
    For any \(T, M\) and any \(g \in \bin^n\)
    \begin{displaymath}
        \probsub{\hgen\sample{\bin^n}}{H_{\hgen}(T, M) \xor \hgen = g} \leq d(T, M)/2^n
    \end{displaymath}
    Proof: \(H(T, M)\) has a zero constant term and
    cannot be equal to the polynomial \(x^n\hpoly\).
    \(H(T, M) \xor g \xor \hgen = H(T, M) \xor g \xor x^n\hpoly\)
    thus cannot be the zero polynomial and has
    degree at most \(d(T, M)\).
\end{description}

\subsection{H-coefficient technique}\label{hco}
The H-coefficient technique was introduced by Patarin in 1991~\cite{ppdes,hco}.
We highly recommend the exposition
of~\cite{hco2} Section 3,
``The H-coefficient Technique in a Nutshell'';
we here present a simpler exposition that
does not cover the technique in its full
generality but only our use of it.
Our use of the symbol \(\mathcal{T}\) and the term
``compatible'' differ from~\cite{hco2}.

We wish to bound the adversary's ability to distinguish between
two ``worlds'', world X (the ``real world'') and world Y (the ``ideal world'').
Each world is a probability distribution over
deterministic oracles the adversary interacts with.

We consider only deterministic adversaries.
A randomized adversary can be considered as a random draw
from a population of deterministic adversaries, so
a bound on the advantage achievable by a deterministic
adversary bounds the whole population and therefore
the advantage of the randomized adversary. In what follows
we consider the adversary \(A\) fixed; only the world, and the
particular oracles drawn from that world, vary.

When the adversary interacts with the oracle,
a transcript \(\tau\) of queries and responses is created.
\(\Tc\) is the set of ``compatible transcripts'':
if \(\tau \in \Tc\) then for the fixed adversary,
there is some oracle
that results in its creation. For example,
since the adversary is deterministic, the first query
will always be the same; a transcript that
does not start with this query is not a compatible transcript.
For a given \(\tau \in \Tc\),
a deterministic adversary must always
return the same answer; call this answer \(A(\tau)\).

Let random variables \(X\) and \(Y\)
represent the distribution of transcripts
in world X and world Y respectively, so that
each transcript \(\tau\) has a probability \(\prob{X = \tau}\)
of arising in world X, and similarly \(\prob{Y = \tau}\) in world Y\@.
The adversary's distinguishing advantage is then
\(\left|\prob{A(Y) = 1} - \prob{A(X) = 1}\right|\).
Without loss of generality,
we assume that \(\prob{A(Y) = 1} \geq \prob{A(X) = 1}\).
We further assume that \(A(\tau)\) is optimal:
\(A(\tau) = 1\)
when \(\prob{Y = \tau} > \prob{X = \tau}\) and 0 otherwise.

In \autoref{mainlemma}, we partition \(\Tc\) into \(\Tg\) and \(\Tb\),
and prove that:
\begin{itemize}
    \item \(\prob{Y = \tau} \leq \prob{X = \tau}\) for all \(\tau \in \Tg\)
    \item \(\prob{Y \in \Tb} \leq \epsilon\)
\end{itemize}
It follows that \(A(\tau) = 0\) for all \(\tau \in \Tg\),
and therefore that \(\prob{A(Y) = 1} \leq \epsilon\),
from which we bound the distinguishing advantage:
\(\prob{A(Y) = 1} - \prob{A(X) = 1} \leq \epsilon\).

With this technique, only the first proof
need consider the probability
distribution of world X at all,
and this proof need only consider good transcripts.
The bulk of the work, proving
\(\prob{Y \in \Tb} \leq \epsilon\),
involves only world Y, which is far simpler
to reason about.

\subfile{mainlemma.tex}
\subsection{Security bound}\label{securitybound}
By a standard substitution argument~\cite{cbcsec,concrete} we have that
\begin{displaymath}
    \advantage{\HCTR[\Perm(n)]}{\HCTR[E]}[(q, \sigma, t)]
    \leq \advantage{\pm \mathrm{prp}}{E}[(\sigma + 2, t + \sigma t')]
\end{displaymath}
where \(t'\) is a small constant
representing the per-block cost of simulating HCTR2, and
\(\sigma + 2\) bounds the number of block cipher calls made by the simulator.

Halevi and Rogaway's PRP-RND lemma
\cite[Appendix C, Lemma 6]{cmc} tells us that
\begin{displaymath}
    \advantage{\pm \widetilde{\mathrm{prp}}}{\pm\widetilde{\mathrm{rnd}}}[(q, \sigma, t)] 
    \leq \left.\binom{q}{2}\middle/2^n\right.
    \leq q^2/2^{n+1}
\end{displaymath}
Putting these together with our main lemma, we conclude
\begin{align*}
    &\advantage{\pm \widetilde{\mathrm{prp}}}{\HCTR[E]}[(q, \sigma, t)] \\
    \leq & \quad \advantage{\pm \widetilde{\mathrm{prp}}}{\pm\widetilde{\mathrm{rnd}}}[(q, \sigma, t)] \\
    & + \advantage{\pm\widetilde{\mathrm{rnd}}}{\HCTR[\Perm(n)]}[(q, \sigma, t)] \\
    & + \advantage{\HCTR[\Perm(n)]}{\HCTR[E]}[(q, \sigma, t)] \\
    \leq & \quad \advantage{\pm \mathrm{prp}}{E}[(\sigma + 2, t + \sigma t')] \\
    &+ \left.\left(3\sigma^2 + 2q\sigma + q^2 + 7\sigma + 2\right)\middle/2^{n+1}\right.
\end{align*}
\end{document}
