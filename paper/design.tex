% Copyright 2021 Google LLC
%
% Use of this source code is governed by an MIT-style
% license that can be found in the LICENSE file or at
% https://opensource.org/licenses/MIT.

%!BIB program = biber
%!TeX program = lualatex
%!TeX spellcheck = en-US

\documentclass[hctr2.tex]{subfiles}
\begin{document}
\section{Design of HCTR2}\label{design}
HCTR2 is intended as a successor to HCTR and retains several of the
features that make it an attractive design:
\begin{itemize}
    \item An unbalanced Feistel-like network based on universal
    hashing, with a single block encryption on the narrow side.
    This gives excellent performance and parallelizability,
    as well as natural handling of messages that are not
    a multiple of the block size.
    It also efficiently handles messages as small as the
    block size; our work on HCTR2 is motivated by
    filename encryption for Linux's
    \emph{fscrypt} module\cite{fscrypt}, where
    short messages will be commonplace.
    \item Use of \(\MM \xor \UU\) in generating \(S\), which means
    that an adversary's control over \(S\) is very limited for both
    encryption and decryption queries; this is used in the
    proof of security to bound \(S_i^r \qeq S_j^s\)
    and avert a cubic term in the security bound.
    \item The CTR mode variant XCTR\@.
    Because of the extra constant \(L\), it would be straightforward
    to prove secure an HCTR2 variant that used CTR mode.
    However, unlike CTR, XCTR
    never needs to maintain a counter larger
    than needed for the message size;
    when a 128-bit nonce is used, CTR 
    must use a 128-bit counter, and is
    at risk of implementations
    whose flaws only manifest on the
    rare occasions that a counter overflows
    into the next word.
    GCM\cite{gcm} uses a variant of CTR
    in which only 32 bits are incremented
    for the same reason;
    XCTR seems a more elegant solution.
\end{itemize}

HCTR2 differs from HCTR in the following ways:

\begin{itemize}
    \item We introduce the extra
    key-dependent constant \(L\),
    so that where HCTR has \(S = \MM \xor \UU\)
    we have \(S = \MM \xor \UU \xor L\),
    fixing the issue described in \autoref{badproof}
    and restoring the quadratic security bound.
    \item We redesign the polynomial hash input format, as described in
    \autoref{inputformatting}, fixing the issue described in \autoref{badhash}.
    \item We accept a variable-length tweak.
    This increases flexibility,
    and eliminates the risk of attacks where
    two users of the same key have different ideas
    of what the tweak length is.
    \item We derive \(\hgen \gets E_k(\fromint(0))\)
    and \(L \gets E_k(\fromint(1))\) from the block cipher key;
    this makes HCTR2 more convenient to use.
    \item We specify POLYVAL\cite{aes_gcm_siv,aes_gcm_siv_rfc}
    as the polynomial evaluation function,
    for reasons set out in \autoref{choosingpolyval}.
    \item We present a new proof, based on the H-coefficient technique,
    with a tighter bound.
    \item We specify endianness and the like so that implementations can be interoperable. We use little-endian representation everywhere,
    since this is faster on nearly all modern platforms.
    This is another difference between XCTR and CTR, since
    CTR is defined to be big-endian.
    \item We rename some variables in our exposition and proof to allow some more standard usage.
    \item We provide a sample implementation and test vectors.
\end{itemize}

\subsection{Comparison of SPRP modes}
We considered a number of modes aiming to provide
tweakable super-pseudorandom permutations
as the basis for our design before settling on HCTR\@.
HCTR is simpler than all of these modes except HHFHFH;
each also has specific qualities that led us to choose
an HCTR variant in preference.
\begin{itemize}
    \item CMC\cite{cmc}, EME\cite{eme}, and EME*\cite{emestar}
    require two block cipher calls per input block.
    \item PEP\cite{pep}, TET\cite{tet}, and HEH\cite{heh} are 
    complex, and are either
    unable to handle messages that are not multiples of the
    block size, or require extra ciphertext-stealing like tricks
    to handle such messages.
    In addition, they require five passes over the data,
    or three if passes are combined.
    Thanks to the simplicity of the unbalanced Feistel network,
    HCTR and HCTR2 require three passes, or two if combined.
    \item HCH\cite{hch} is similar to HCTR
    but uses \(S = E_k(\MM \xor \UU)\).
    With this change the authors were able to prove a
    quadratic security bound. Our modification,
    \(S = \MM \xor \UU \xor E_k(\fromint(1))\), saves
    a block cipher call per invocation.
    \item HSE\cite{hse} achieves similar performance to
    this mode, but is significantly more complicated,
    and accepts only an \(n\)-bit tweak.
    \item HMC\cite{hmc} allows the encryption of the
    first block to run in parallel with subsequent blocks,
    but at a cost of significant complication of decryption,
    which does not gain this advantage;
    at key setup time, the multiplicative inverse of
    the hash key must be calculated.
    In addition, like HCTR (\autoref{badhash}) HMC's hash
    function is not correctly injective onto polynomials.
    \item FAST\cite{fast} uses only the encryption
    direction of the block cipher. However 
    it is fairly complex, and the minimum
    message size is twice the width of the block cipher;
    for our application we need efficient handling of small messages.
    \item HHFHFH\cite{hufflehuff} is a particularly
    clean design based on a four-round Feistel network,
    but requires a \(2^{4n}\)-bit message size for
    \(n\)-bit security; again this doesn't meet our
    small-message needs.
\end{itemize}

\subsection{Hash function design}\label{hashdesign}

\subsubsection{POLYVAL}\label{choosingpolyval}

We aim to specify HCTR2 in sufficient detail
for implementations to be interoperable,
so we must be precise about endianness and the like
in \(\GF(2^{128})\) polynomial evaluation.
The most widely used  convention is that of GCM's GHASH\cite{gcm}.
However, GHASH is not consistent in its endianness conventions,
which increases implementation complexity and reduces efficiency.

Instead, we use the POLYVAL function
defined in \cite{aes_gcm_siv,aes_gcm_siv_rfc}.
POLYVAL incurs a small cost in specification and proof complexity because 
the polynomial is evaluated not at the parameter \(\hgen\)
but at \(x^{-n}\hgen\)
so that Montgomery multiplication\cite{montmul} can be key-agile.
However it is carefully designed, efficient
on processors with carryless multiply instructions
(1.2x faster than GHASH according to \cite{aes_gcm_siv})
and offers an efficient conversion between POLYVAL and GHASH hashing
which allows code/hardware for one to be used for the other.

\subsubsection{Input formatting}\label{inputformatting}

The formatting of inputs to the polynomial hash in HCTR2 is significantly
different from that in HCTR\@. Our design goals are:

\begin{itemize}
    \item fix the flaw described in \cite{kumarhctr}
    \item allow a variable-length tweak
    \item guarantee \(H(T, M) \neq \hgen\), required
    because \(\hgen \gets E_k(\fromint(0))\)
    \item allow implementations to precompute as much as possible, to reduce \(\GF(2^n)\) multiplications
\end{itemize}

See \autoref{hproperties}
for the properties we require of the hash function.

To fix the flaw described in \cite{kumarhctr},
we eliminate the zero-length special case
by adding one to the length before encoding it.

We process the tweak before the message, so that implementations need only
process the tweak once for each encryption/decryption, instead of twice.

With the introduction of the constant \(L\)
our security proof no longer relies on the hash function
having property 2 of \cite[Section~3.3]{hctr}.
This allows us to move the length block first, so
that implementations need only process it once per encryption/decryption.
It is never zero, so its position in the
polynomial can be inferred from the degree.

This change means that \(\ceil{\abs{T}/n} + \ceil{\abs{M}/n}\)
can be inferred from the degree of the polynomial.  
If we append a \(\bini\) to the message before padding with zeroes,
we need only encode only the tweak length in the length block,
and the message length can then be inferred.
For users whose tweaks are always the same length
this means the length block is always the same, 
so the multiplication with \(\hpoly\) can be
precomputed.

However, in the very common case where 
the message length is a multiple of the block size,
we don't want an extra multiplication
for an extra block containing only the appended \(\bini\) bit.
Borrowing from~\cite{xcbc},
we don't append a \(\bini\) bit to such messages.
Instead, we indicate whether
the message length is a multiple of the block size
in the least significant bit of the length block.
If all tweaks are of length \(t\), implementations can cache
\(\fromint(2t + 2)\hpoly\) and \(\fromint(2t + 3)\hpoly\)
and use one of these to start hashing as appropriate,
XORing this value directly with the first block of the tweak.

\subsubsection{Alternatives considered}

Like HCTR, HCTR2's almost-XOR-universal hash function uses a standard polynomial
evaluation in $\GF(2^n)$. This uses an $n$-bit key and requires $l$ field
multiplications where $l$ is the number of blocks.
We considered several alternatives:

\emph{BRW polynomials}: BRW polynomials\cite{pema}\cite{heh2} are theoretically
attractive since they only need \(\floor{l/2}\) multiplications
to evaluate. However, they pose a number of difficulties. \cite{brweval} gives a
nonrecursive algorithm that handles variable-length messages, but it is complex,
uses temporary space that grows logarithmically with the message length,
and does not handle incremental computation well.
Standard polynomials avoid these issues;
fast and correct implementations are easier to write,
and implementers have much more control over
code size, precomputation, instruction-level
parallelism, number of reductions, and so forth.
Finally, preserving our guarantees of injectivity on variable-length
tweaks and messages, and the other hash function properties we need
to guarantee, proved challenging.

\emph{Hash2L}: Hash2L\cite{hash2l} solves two issues with BRW polynomials.
First, it limits the depth of recursion, and thus the space needed, by replacing
the uppermost levels by a simpler Horner based evaluation. This slightly increases
the number of multiplications per block but solves several implementation issues.
Secondly, it adds an extra multiplication at the end to include length information
so that the whole construction is injective on variable-length messages.
Where most messages are large, such as for disk encryption, a variant of
HCTR2 that uses Hash2L could be attractive; however since performance
on small messages is key to our application we prefer the simplicity
and optimization potential of Horner evaluation.

\emph{Polynomials over non-binary fields}: When CPU instructions for carryless
multiplication are unavailable, hashes using non-binary fields such as
Poly1305\cite{poly1305} tend to be faster than hashes using binary fields.
However, HCTR2 primarily targets
processors that support carryless multiplication, 
and on such processors hashes
using binary fields tend to be faster and simpler.

\emph{Multivariate hashes}: Adiantum\cite{adiantum} builds an
almost-$\Delta$-universal hash function using the multivariate hash
NH\cite{umac1} combined with polynomial evaluation. Where NH is faster than
polynomial evaluation, this increases performance. However, this adds
complexity, and NH requires a long key which needs to be derived and cached. 
HCTR2 primarily targets processors where polynomial evaluation is fast, so
we do not add an NH layer.

\end{document}
