\documentclass[aspectratio=169]{beamer}

\usepackage{etoolbox}
\usepackage[logic,probability,advantage,adversary,landau,sets,operators]{cryptocode}
\usepackage{tikz}

\usetikzlibrary{groupops}

\newcommand*{\Concat}{\Vert}
\newcommand*{\defeq}{\stackrel{\text{def}}{=}}
\newcommand*{\hgen}{\bar{h}}
\newcommand*{\hpoly}{h}
\newcommand*{\MM}{\mathit{MM}}
\newcommand*{\qeq}{\stackrel{\text{?}}{=}}
\newcommand*{\Tb}{\mathcal{T}_\mathrm{bad}}
\newcommand*{\Tc}{\mathcal{T}_\mathrm{c}}
\newcommand*{\Tg}{\mathcal{T}_\mathrm{good}}
\newcommand*{\UU}{\mathit{UU}}

\DeclareMathOperator{\fromint}{bin}
\DeclareMathOperator{\GF}{GF}
\DeclareMathOperator{\HCTR}{HCTR2}
\DeclareMathOperator{\pad}{pad}
\DeclareMathOperator{\Perm}{Perm}
\DeclareMathOperator{\POLYVAL}{POLYVAL}
\DeclareMathOperator{\poly}{poly}
\DeclareMathOperator{\XCTR}{XCTR}

\newcommand{\binary}[1]{{\color{red!80!black} \mathtt{#1}}}
\newcommand{\bino}{\binary{0}}
\newcommand{\bini}{\binary{1}}
\renewcommand{\bin}{\{\bino,\bini\}}

\renewcommand{\pcadvantagesuperstyle}[1]{#1}

\newtoggle{oldhctr}
\togglefalse{oldhctr}
\newcommand*{\hdiag}{\hgen}


\newenvironment*{figslide}{
    \begin{columns}
        \begin{column}{.4\textwidth}

}{
\end{column}
\begin{column}{.6\textwidth}
    \begin{figure}
        \subfile{hctr2fig.tex}
    \end{figure}
\end{column}
\end{columns}
}

\setbeamertemplate{navigation symbols}{}

\title{HCTR2}
\author{Paul Crowley, Eric Biggers, Nathan Huckleberry}
\institute{Google LLC}
\date{2023-10-03}

\usepackage{subfiles}
\begin{document}

\begin{frame}
    \begin{figslide}
        \titlepage
    \end{figslide}
\end{frame}

\begin{frame}

\frametitle{Background: Adiantum}

\begin{itemize}
    \item 2018: \textit{Adiantum: length-preserving encryption for entry-level processors}
    \item A wide-block mode
    \item Fast without AES+GHASH instructions
    \item Efficient on 0.5kB-4kB messages
\end{itemize}
\end{frame}

\begin{frame}

    \frametitle{Background: HCTR2}
    
    \begin{itemize}
        \item We needed a new wide-block mode
        \item Fast \textit{with} AES+GHASH instructions
        \item Efficient on short messages (16B-64B)
        \item Secure and fully specified
    \end{itemize}
\end{frame}

\begin{frame}

\frametitle{The field}

\begin{itemize}
    \item Much interest in wide block modes around 2005--2009
    \item Many proposals: CMC, EME, EME*, PEP, TET, HEH, HCH, HSE, HMC\ldots
    \item We chose HCTR (Wang, Feng, and Wu 2005)
\end{itemize}
\end{frame}

\begin{frame}

\frametitle{Hash-encrypt-hash}

\begin{itemize}
    \item Any secure wide-block mode has at least three passes
    \item Almost-universal hashing is faster than encryption
    \item So hash-encrypt-hash constructions are fastest
    \item HHFHFH is hash-encrypt-hash in disguise
\end{itemize}
\end{frame}

\toggletrue{oldhctr}

\begin{frame}

\frametitle{HCTR}

    \begin{figslide}
        \begin{itemize}
            \item Simple
            \item Fast with AES+GHASH instructions
            \item No ciphertext stealing tricks needed
            \item XCTR mode
            \item Tight quadratic security claim
        \end{itemize}
    \end{figslide}
\end{frame}

\begin{frame}
    \frametitle{XCTR mode}
    \begin{columns}
        \begin{column}{.4\textwidth}
            \begin{itemize}
                \item CTR: nonce PLUS counter
                \item XCTR: nonce XOR counter
                \item No 128-bit addition required
                \item No nasty GCM hack
                \item Little-endian
            \end{itemize}
        \end{column}
        \begin{column}{.6\textwidth}
            \begin{align*}
                \operatorname{CTR}_k(S) =& E_k(\fromint(S + 1)) \\
                & \Concat  E_k(\fromint(S + 2)) \\ 
                & \Concat  E_k(\fromint(S + 3)) \Concat \cdots \\
                \XCTR_k(S) =& E_k(S \xor \fromint(1)) \\
                & \Concat  E_k(S \xor \fromint(2)) \\ 
                & \Concat  E_k(S \xor \fromint(3)) \Concat \cdots \\
            \end{align*}
        \end{column}
    \end{columns}
\end{frame}

\begin{frame}

\frametitle{HCTR issues}
\begin{figslide}
        \begin{itemize}
            \item Block-length message special case
            \item Hash is non injective
            \item Error in quadratic security proof
            \item HCTR2 fixes these, and ``sands the edges''
        \end{itemize}
\end{figslide}

\end{frame}

\togglefalse{oldhctr}

\begin{frame}

\frametitle{HCTR2}
\begin{figslide}
    \begin{itemize}
        \item New key-dependent constant \(L\) XORed into \(S\)
        \item Rescues quadratic security bound
    \end{itemize}
\end{figslide}

\end{frame}

\begin{frame}
    \frametitle{HCTR2 hash function}
    \begin{columns}
        \begin{column}{.3\textwidth}
            \begin{itemize}
                \item Fixes encoding to be injective
                \item Handles variable-length tweak
                \item Length+tweak processed only once
                \item Uses POLYVAL for speed
            \end{itemize}
        \end{column}
        \begin{column}{.7\textwidth}
            \begin{align*}
                &\operatorname{encoding}(T, M) =\\
                &\begin{cases}
                    \fromint(2\abs{T} + 2) \Concat \pad(T) \Concat M &
                    \abs{M} = 0 \pmod{n}  \\
                    \fromint(2\abs{T} + 3) \Concat \pad(T) \Concat \pad(M \Concat \bini) &
                    \abs{M} \ne 0 \pmod{n}
                \end{cases}
            \end{align*}
        \end{column}
    \end{columns}
\end{frame}

\begin{frame}

\frametitle{Sanding the edges}
\begin{figslide}
    \begin{itemize}
        \item \(\hgen\), \(L\) derived from block cipher
        \item Endianness, field convention specified
        \item Sample implementation and test vectors
        \item In Linux kernel now
    \end{itemize}

\end{figslide}
\end{frame}

\begin{frame}

    \frametitle{Quadratic security}

    \begin{align*}
        &\advantage{\pm \widetilde{\mathrm{prp}}}{\HCTR[E]}[(q, \sigma, t)] \\
        \leq & \quad \advantage{\pm \mathrm{prp}}{E}[(\sigma + 2, t + \sigma t')] \\
        &+ \left.\left(3\sigma^2 + 2q\sigma + q^2 + 7\sigma + 2\right)\middle/2^{n+1}\right.
    \end{align*}
        

    \begin{itemize}
        \item \(q\) queries, \(\sigma\) blocks, \(t\) time
        \item \(H\)-coefficient based proof
    \end{itemize}
\end{frame}
    
\begin{frame}
\frametitle{Future work: better than quadratic security?}

\begin{itemize}
    \item This is all still speculative
    \item Inspired by AES-GCM-SIV
    \item Per-message keys derived from nonce
    \item Derive \(\hgen\) and \(L\) in the same way
    \item Multi-target security matters if keys are 128-bit
    \item Proof in ideal cipher model
\end{itemize}
\end{frame}

\end{document}