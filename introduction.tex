% Copyright 2021 Google LLC
%
% Use of this source code is governed by an MIT-style
% license that can be found in the LICENSE file or at
% https://opensource.org/licenses/MIT.

%!BIB program = biber
%!TeX program = lualatex
%!TeX spellcheck = en-US

\documentclass[hctr.tex]{subfiles}
\begin{document}
\section{Introduction}
A symmetric encryption construction such as AES-GCM\cite{gcm}
will produce a ciphertext which is several bytes larger than
the plaintext, once the IV and MAC are included. However,
there are several situations such as disk sector encryption
where the ciphertext must be the exact same size as the
plaintext, and what is required is a tweakable
super-pseudorandom permutation\cite{cmc}. 
\cite{adiantum} includes a detailed history of length-preserving encryption.
Where AES and multiplication in \(GF(2^n)\) are hardware-accelerated,
the construction HCTR\cite{hctr} is among the most efficient for 
tweakable length-preserving encryption, using
a single block cipher invocation and two multiplies
per block. In this paper, we present a revised version, HCTR2.
This revision addresses the following issues:
\begin{itemize}
    \item \cite{kumarhctr} observes that the HCTR polynomial hash function
    is not \(\epsilon\)-almost-XOR-universal\cite{eadu} as claimed.
    HCTR2's hash function fixes this property.
    \item An error in the proof presented in \cite{hctrquad}
    invalidates the claimed security bound; with our revised
    mode we can claim a tighter bound.
    \item HCTR2 supports using tweaks of any length with a single key.
    \item HCTR's key is a block cipher key, plus an \(n\)-bit
    hash key. HCTR2's key is simply
    the block cipher key.
    \item We modify the hash function to allow more precalculation
    for greater performance.
    \item We specify endianness and the like for interoperability.
    \item We provide a sample implementation and test vectors.
\end{itemize}

We know of no patents affecting use of HCTR2.
We begin with the specification and proof for HCTR2, laying
the groundwork for describing the issues with HCTR (\autoref{hctrissues}),
after which we discuss HCTR2 design decisions
and report on its implementation on x86-64 and ARM64.
\end{document}
