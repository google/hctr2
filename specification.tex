% Copyright 2021 Google LLC
%
% Use of this source code is governed by an MIT-style
% license that can be found in the LICENSE file or at
% https://opensource.org/licenses/MIT.

%!BIB program = biber
%!TeX program = lualatex
%!TeX spellcheck = en-US

\documentclass[hctr.tex]{subfiles}
\begin{document}
\section{Specification}\label{specification}
\subsection{Notation}
\begin{itemize}
    \item $\abs{X}$: length of $X \in \bin^{*}$ in bits
    \item $\lambda$: the empty string $\abs{\lambda} = 0$
    \item $X[a;l]$: the substring of $X$ of length $l$ starting at the 0-based index $a$
    \item $\Concat$: bitstring concatenation
    \item \(\xor\): bitwise XOR
    \item \(n\): block size in bits
    \item $\fromint_l: \{0 \ldots 2^l-1\} \rightarrow \bin^l$:
    little-endian conversion of integers to binary; 
    \(\fromint(x)\) means \(\fromint_n(x)\)
    \item $\pad(X) = X \Concat 0^v$
    where $v$ is the least integer $\geq 0$ such that $n$ divides $\abs{X} + v$
    \item \(x, x^2, \ldots\): elements of the finite field \(\GF(2^n)\)
    \item \(E: \mathcal{K} \times \bin^n \rightarrow \bin^n\): 
    \(n\)-bit block cipher with keyspace \(\mathcal{K}\);
    our concrete proposal uses AES\cite{aes},
    so \(n=128\) and \(\mathcal{K}\) is
    \(\bin^{128}\), \(\bin^{192}\), or \(\bin^{256}\)
    \item \(\mathcal{T}\): the set of permissible tweaks
    \(\mathcal{T} = \bigcup_{i \in \{0\ldots2^{n-1}-2\}}\bin^i\)
    \item \(\mathcal{M}\): the set of permissible messages
    \(\mathcal{M} = \bin^n \times \mathcal{T} = \bigcup_{i \in \{n\ldots n + 2^{n-1}-2\}}\bin^i\)
\end{itemize}
We map bytes to bitstrings with \(\fromint_8\). Subscripts may denote partial application; if we define $f: A \times B \rightarrow C$ and
$a \in A$ then $f_a: B \rightarrow C$, and if $f_a^{-1}$ exists then $f_a^{-1}(f_a(b)) = b$.

\subsection{Polynomial hash function}\label{hashspec}
We interpret \(n\)-bit blocks as little-endian field elements of \(\GF(2^n)\),
so \(001 \Concat 0^{n-3}\) is interpreted as the element \(x^2\).
Per \cite{aes_gcm_siv} we define
\begin{align*}
    \POLYVAL(h, \lambda) & = 0^n\\
    \POLYVAL(h, A \Concat B) & = (\POLYVAL(h, A) \xor B) \otimes h \otimes x^{-n}
\end{align*}
where \(\abs{B} = n\) and
\(\otimes\) is multiplication in the finite field.
In our concrete proposal, \(n=128\), we reduce by
\(x^{128} + x^{127} + x^{126} + x^{121} + 1\),
and the value of the field element \(x^{-n}\)
is equal to \(x^{127} + x^{124} + x^{121} + x^{114} + 1\).

For hash key \(h \in \bin^n\), tweak \(T\) and message \(M\), we define:
\begin{align*}
    & H_h(T, M) \\
    \defeq & 
    \begin{cases}
        \POLYVAL(h, \fromint(2\abs{T} + 2) \Concat \pad(T) \Concat M) &
        \text{if } n \text{ divides } \abs{M} \\
        \POLYVAL(h, \fromint(2\abs{T} + 3) \Concat \pad(T) \Concat \pad(M \Concat 1)) &
        \text{otherwise}
    \end{cases}
\end{align*}

\subsection{XCTR mode}
HCTR and HCTR2 use an unusual mode of stream encryption,
which we here name \emph{XCTR mode}:
\begin{displaymath}
    \XCTR_k(S) = E_k(S \xor \fromint(1)) \Concat E_k(S \xor \fromint(2)) \Concat E_k(S \xor \fromint(3)) \Concat \cdots
\end{displaymath}
Thus, generating the first \(m\) bits \(\XCTR_k(S)[0; m]\) takes \(\ceil{m/n}\) block cipher calls. 

\subsection{HCTR2 encryption}
From these definitions, HCTR2 is as defined in \autoref{pseudocode}.
\subfile{algorithm.tex}
\end{document}
